%% ++++++++++++++++++++++++++++++++++++++++++++++++++++++++++++
%% Hauptdatei, Wurzel des Dokuments
%% ++++++++++++++++++++++++++++++++++++++++++++++++++++++++++++

% Headerfeld, Typ des Dokumentes, einzubindende Packages.
% Hier bei Bedarf Änderungen vornehmen.
\documentclass
[   twoside=false,     % Einseitiger oder zweiseitiger Druck?
    fontsize=12pt,     % Bezug: 12-Punkt Schriftgröße
    DIV=15,            % Randaufteilung, siehe Dokumentation "KOMA"-Script
    BCOR=17mm,         % Bindekorrektur: Innen 17mm Platz lassen. Copyshop-getestet.
%    headsepline,
    headsepline,  % Unter Kopfzeile Trennlinie (aus: headnosepline)
    footsepline,  % Über Fußzeile Trennlinie (aus: footnosepline)
    open=right,        % Neue Kapitel im zweiseitigen Druck rechts beginnen lassen
    paper=a4,          % Seitenformat A4
    abstract=true,     % Abstract einbinden
    listof=totoc,      % Div. Verzeichnisse ins Inhaltsverzeichnis aufnehmen
    bibliography=totoc,% Literaturverzeichnis ins Inhaltsverzeichnis aufnehmen
    titlepage,         % Titelseite aktivieren
    headinclude=true,  % Seiten-Head in die Satzspiegelberechnung mit einbeziehen
    footinclude=false, % Seiten-Foot nicht in die Satzspiegelberechnung mit einbeziehen
    numbers=noenddot   % Gliederungsnummern ohne abschließenden Punkt darstellen
]   {scrreprt}         % Dokumentenstil: "Report" aus dem KOMA-Skript-Paket

\usepackage[active]{srcltx}
%\usepackage[activate=normal]{pdfcprot} % Optischer Randausgleich -> pdflatex!
\usepackage{ifthen}
\usepackage[ngerman]{babel}   % Neue Deutsche Rechtschreibung
%\usepackage[latin1]{inputenc} % Zeichencodierung nach ISO-8859-1
\usepackage[utf8]{inputenc}   %	Zeichencodierung nach UTF-8 (Unicode)
\usepackage[T1]{fontenc}
%\usepackage{ae} % obsolet und durch lmodern ersetzt
\usepackage{lmodern}
\usepackage[T1]{url}
\usepackage[final]{graphicx}
\RequirePackage{scrlfile}
\ReplacePackage{scrpage2}{scrlayer-scrpage}
% old: \usepackage[automark]{scrpage2}
\usepackage[automark]{scrlayer-scrpage}
\usepackage{setspace}
%\usepackage[first,light]{draftcopy} % Für Probedruck
\usepackage[plainpages=false,pdfpagelabels,hypertexnames=false]{hyperref}

% Tiefe der Kapitelnummerierung beeinflussen
\setcounter{secnumdepth}{3} % Tiefe der Nummerierung
\setcounter{tocdepth}{3}    % Tiefe des Inhaltsverzeichnisses

% Datum anpassen
\newcommand{\leadingzero}[1]{\ifnum #1<10 0\the#1\else\the#1\fi}    
\renewcommand{\today}{\leadingzero{\day}.\leadingzero{\month}.\the\year}     % DD.MM.YYYY

% Hier in die zweite geschweifte Klammer jeweils
% die persönlichen Daten und das Thema der Arbeit eintragen:
\newcommand{\artderausarbeitung}{C - Trainer Arbeit}
\newcommand{\namedesautors}{Jonathan Skopp}
\newcommand{\themaderarbeit}{Starke und schwache Felder}

% PDF Metadaten definieren
\hypersetup{
   pdftitle={\themaderarbeit},
   pdfsubject={\artderausarbeitung},
   pdfauthor={\namedesautors},
   pdfkeywords={\artderausarbeitung; TU-Ilmenau}}


% Abkürzungsverzeichnis beeinflussen. Hier nichts ändern!
\usepackage[intoc]{nomencl}
  \AtBeginDocument{\setlength{\nomlabelwidth}{.25\columnwidth}}
  \let\abbrev\nomenclature
  \renewcommand{\nomname}{Abkürzungsverzeichnis und Formelzeichen}
  \renewcommand{\nomlabel}[1]{#1 \dotfill}
  \setlength{\nomitemsep}{-\parsep}
  \makenomenclature

\usepackage[normalem]{ulem}
  \newcommand{\markup}[1]{\textbf{#1}}

% Seitenlayout festlegen. Hier nichts ändern!
\pagestyle{scrplain}
\ihead[]{\headmark}
\ohead[]{\pagemark}
\chead[]{}
\ifoot[]{}
\ofoot[]{\scriptsize \artderausarbeitung\ \namedesautors}
\cfoot[]{}
\renewcommand{\titlepagestyle}{scrheadings}
\renewcommand{\partpagestyle}{scrheadings}
\renewcommand{\chapterpagestyle}{scrheadings}
\renewcommand{\indexpagestyle}{scrheadings}

% Abschnittsweise Nummerierung anstatt fortlaufend. Hier nichts ändern!
\makeatletter
\@addtoreset{equation}{chapter}
\@addtoreset{figure}{chapter}
\@addtoreset{table}{chapter}
\renewcommand\theequation{\thechapter.\@arabic\c@equation}
\renewcommand\thefigure{\thechapter.\@arabic\c@figure}
\renewcommand\thetable{\thechapter.\@arabic\c@table}
\makeatother

% Quelltextrahmen, klein. Hier nichts ändern!
\newsavebox{\inhaltkl}
\def\rahmenkl{\sbox{\inhaltkl}\bgroup\small\renewcommand{\baselinestretch}{1}\vbox\bgroup\hsize\textwidth}
\def\endrahmenkl{\par\vskip-\lastskip\egroup\egroup\fboxsep3mm%
\framebox[\textwidth][l]{\usebox{\inhaltkl}}}

% Quelltextrahmen, normale Groesse. Hier nichts ändern!
\newsavebox{\inhalt}
\def\rahmen{\sbox{\inhalt}\bgroup\renewcommand{\baselinestretch}{1}\vbox\bgroup\hsize\textwidth}
\def\endrahmen{\par\vskip-\lastskip\egroup\egroup\fboxsep3mm%
\framebox[\textwidth][l]{\usebox{\inhalt}}}

% Trennvorschläge für falsch getrennte Wörter.
% Wird häufig bei eingedeutschen Wörtern benötigt, da LaTeX hierbei
% gerne falsch trennt. Alternativ kann auch im Fliesstext ein
% Trennvorschlag per "\-" hinterlegt werden, bspw.:
% Die Hard\-ware besteht aus A und B.
\hyphenation{
Hard-ware
}

% Sonstige Befehlsdefinitionen hier ablegen.
\newcommand{\entspricht}{\stackrel{\wedge}{=}}

% Tabellenspaltendefinitionen mit fester Breite --> somit Zeilenumbruch innerhalb einer Zelle möglich
% aus http://www.torsten-schuetze.de/tex/tabsatz-2004.pdf
\usepackage{array, booktabs}
\newcolumntype{f}{>{$}l<{$}}
\newcolumntype{n}{>{\raggedright}l}
\newcolumntype{N}{>{\scriptsize}l}
\newcolumntype{v}[1]{>{\raggedright\hspace{0pt}}m{#1}}
\newcolumntype{V}[1]{>{\scriptsize\raggedright\hspace{0pt}}m{#1}}
\newcolumntype{Z}[1]{>{\raggedright\centering}m{#1}}
\newcolumntype{k}[1]{>{\raggedright}p{#1}}
% ergibt Tabllenspalte fester Breite, linksbündig
% Umbruch innerhalb der Zelle mit \\, neue Tabellezeile mit \tabularnewline
% \addlinespace für Gruppentrennung (aus \texttt{booktabs.sty})


\begin{document}
\onehalfspacing
\begin{titlepage}
\centering
\includegraphics[scale=0.5]{DSB.png}\\[3ex]
{\Large \textsc{Grundlagen Training }}\\[3ex]
{\Large Schach}\\[3ex]
\vfill
{\Large \textbf{\artderausarbeitung}}\\[4ex]
{\large \textbf{\themaderarbeit}}\\[5ex]
\vfill
\begin{tabular}{rl}
\hline \\
Person:              & \quad \namedesautors\\[1,5ex]
Start:               & \quad 08.\,12.\,2020\\[1,5ex]
Ausbildung:          & \quad Mein Schachtraining\\[1,5ex]
Trainer:             & \quad WGM Josefine Heinemann\\[1,5ex]

\end{tabular}
\vfill
\end{titlepage}








%% ++++++++++++++++++++++++++++++++++++++++++++++++++++++++++++
%% Zusammenfassung, Abstract
%% ++++++++++++++++++++++++++++++++++++++++++++++++++++++++++++


\renewcommand{\abstractname}{Kurzfassung}
\begin{abstract}
\ldots \emph{Hier später die eigene deutsche Kurzfassung einfügen}\ldots
\par{}
Dieses Dokument soll als Gerüst für eigene \LaTeX\ Dokumente dienen und
gleichzeitig Beispiele für häufig verwendete Konstrukte wie Tabellen,
Formeln oder Grafiken liefern. Es empfielt sich, diese Elemente per
Cut\&Paste zu kopieren und einzufügen.
\end{abstract}




% Inhaltsverzeichnis
\cleardoublepage % Seitenumbruch erzwingen vor Änderung des Nummerierungsstils
\pagenumbering{roman} % Nummerierung der Seiten ab hier: i, ii, iii, iv...
\pagestyle{scrheadings} % Ab hier mit Kopf- und Fusszeile
\tableofcontents

% Die einzelnen Kapitel
\cleardoublepage % Seitenumbruch erzwingen vor Änderung des Nummerierungsstils
\pagenumbering{arabic} % Nummerierung der Seiten ab hier: 1, 2, 3, 4...





%%% Inhalt %%%%

\input{starke_Felder.tex}
\input{schwache_Felder.tex}
\input{Training.tex}
\input{Partien.tex}
\input{Aufgaben.tex}
\input{Loesungen.tex}



%%% Ende %%%

\appendix
\part*{Anhang}




% Literaturverzeichnis einbinden
%% ++++++++++++++++++++++++++++++++++++++++++++++++++++++++++++
%% Anhang: Literaturverzeichnis
%% ++++++++++++++++++++++++++++++++++++++++++++++++++++++++++++


% Mit dem Befehl \nocite werden auch nicht im Text zitierte
% aus der Literaturdatenbank mit in das Literaturverzeichnis aufgenommen.
% Ein "\nocite{*}" übernimmt ungeprüft die komplette Datenbank.
%\nocite{*}

\cleardoublepage
\ihead[]{Literaturverzeichnis}
\bibliographystyle{alphadin}
\bibliography{literatur} % "literatur.bib" ist hier die einzige Literaturdatenbank.

% Alternativ: Mehrere Datenbanken verwenden, falls eine
% oder mehrere umfangreiche Sammlungen exisitieren:
%\bibliography{literatur_buecher,literatur_weblinks}


% Abbildungsverzeichnis einbinden
%% ++++++++++++++++++++++++++++++++++++++++++++++++++++++++++++
%% Anhang: Abbildungsverzeichnis
%% ++++++++++++++++++++++++++++++++++++++++++++++++++++++++++++


% Keine Änderungen vornehmen!
\cleardoublepage
\ihead[]{Abbildungsverzeichnis}
\listoffigures


% Tabellenverzeichnis einbinden
%% ++++++++++++++++++++++++++++++++++++++++++++++++++++++++++++
%% Anhang: Tabellenverzeichnis
%% ++++++++++++++++++++++++++++++++++++++++++++++++++++++++++++

% Hier keine weiteren Änderungen vornehmen
\cleardoublepage
\ihead[]{Tabellenverzeichnis}
\listoftables



% Abkürzungsverzeichnis einbinden
%% ++++++++++++++++++++++++++++++++++++++++++++++++++++++++++++
%% Anhang: Abkürzungsverzeichnis
%% ++++++++++++++++++++++++++++++++++++++++++++++++++++++++++++


% Hier keine weiteren Änderungen vornehmen
\cleardoublepage
\ihead[]{Abkürzungsverzeichnis und Formelzeichen}
\printnomenclature





% Abschlusserklärung
\input{abschlusserklaerung.tex}

\end{document}

