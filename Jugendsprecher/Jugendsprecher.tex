\documentclass[letterpaper,12pt]{article}
\usepackage{tabularx} 
\usepackage{amsmath}  
\usepackage{graphicx} 
\usepackage[margin=1in,letterpaper]{geometry} 
\usepackage[final]{hyperref} 
\hypersetup{
	colorlinks=true,       
	linkcolor=blue,        
	citecolor=blue,        
	filecolor=magenta,     
	urlcolor=blue         
}
\usepackage{blindtext}
%++++++++++++++++++++++++++++++++++++++++

\newcommand{\leadingzero}[1]{\ifnum #1<10 0\the#1\else\the#1\fi}
\renewcommand{\today}{\leadingzero{\day}.\leadingzero{\month}.\the\year}     % DD.MM.YYYY

\begin{document}

\title{Jugendsprecher der Thüringer Schachjugend}
\author{Jonathan Skopp}
\date{\today}
\maketitle

\section{Zur Person}
Mein Name ist Jonathan Skopp und bin 22 Jahre alt und spiele Schach, solange ich denken kann. Mein Vater weckte schon in sehr jungen Jahren mein Interesse für den Sport. In meiner Abiturzeit spielte ich für den SV Grün-Weiß Straußfurt, kurz darauf wechselte ich zu SV Empor Erfurt e. V., wo ich noch heute spiele. Als Trainer und Betreuer habe ich schon Kinder und Jugendliche betreut. Zum Beispiel bei der Deutsche Schul-Mannschaft Meisterschaft 2018 in Friedrichroda.


\section{Ambitionen}

Corona ist das Wort der Stunde. Dies trifft auch den Schach Sport. Durch die Pandemie hat sich das Schach maßgeblich gewandelt. Sehr viele neue Leute freunden sich mit dem königlichen Spiel an. Der Sektor Breitensport ist dermaßen gewachsen. Dazu tragen auch die Streams von Hikaru, TBG und Niclas bei. Jugendtraining und Events zu koordinieren und umzusetzen sind Teil der Aufgaben, die nun anstehen. Turniere mit echten Brettern lassen sich dort nur bedingt einbinden, desto wichtiger ist es, den digitalen Raum für das Schachspiel zu ebnen.
Lichess und Discord bieten dort sehr gute Grundlagen. Die nationale Jugendarbeit (z.B. am Juko) ist immer sehr spannend und bringt auch Events und Wissenswertes für alle. Ich möchte da an das Projekt Bananenturnier und Bar-Camp (ist aktuell in Planung) erinnern. Ich selber spiele gerne Schach, sowohl digital als auch analog. Die Arbeit mit Kindern sowie die Planung und Ausführung von Trainings wie Events bereiten mir sehr große Freude. Deswegen hoffe ich auf eine gute Zusammenarbeit.


\section{Ziele}

\subsection{Allgemein}

Für mich als Jugendwart ist es extrem wichtig, eine Ansprech- und Organisationsperson zu sein. Jeder, egal wer und aus welchem Verein kann Fragen und Ideen einreichen. Wie oben schon beschrieben, hält uns Corona in Atem. Deswegen ist eine Digitalisierung mehr als wichtig. Dadurch kann die ThSj vereinsübergreifende Turniere, Training und sonstige Events veranstalten. Schach erlebt aktuell eine zweite Renaissance. Die aktuelle Frage ist, wie wir als Verbund uns dieser Welle stellen und diese verwalten.


\subsection{Breitensport}

Der Schwerpunkt des Thüringer Schach Sports liegt im Breitensport. Dieser Punkt gliedert sich für mich in zwei große Bereiche.

\subsubsection{Kooperation}
Kooperation ist das A und $\Omega$. Schach ist ein Hobby. Es ist nicht an irgendwelche Vereinsgrenzen oder Landesgrenzen gebunden. Aktuell arbeitet jeder Bezirk für sich. Ein Austausch von Ideen, Trainingsmethoden ist hier sehr erstrebenswert.

\subsubsection{Events}
Events sind ein weiterer wichtiger Bestandteil eines gesundes Verbandslebens. Diese müssen nicht immer nur klassisch und bekannt sein. GM Sebastian Siebrecht macht dies gut vor. Er veranstaltet das Event "Faszination Schach". Dieses kann uns gut eine Vorlage sein, dadurch spricht er viele Leute an, die sonst keine Berührungspunkte mit Schach haben. Er spielte bei der Siegerehrung der DJEM eine Partie Kondi-Blitz gegen die U14w. Solche Events lockern die Schachwelt stark auf. Auf klassische Events darf an dieser Stelle auf keinem Fall verzichtet werden. Jugendturniere wie das Sternturnier oder das Jugendopen sind wertvolle Bestandteile der Thüringer Schachwelt. 

\subsection{Leistungssport}

Manche Schachspieler sehen in Schach mehr als nur ein Hobby. Sie sehen darin ein Sport und behandeln das Schach auch als solches. Neben dem Kader-Programm sollte jedem Spieler, der das nötige Engagement und den Willen dazu hat, eine weitere Förderung angeboten werden. Wie diese aussieht muss an anderer Stelle besprochen werden. Meiner Meinung nach sollte jeder der möchte und dies zeigt, von der Schachjugend die volle Unterstützung erhalten. Dabei ist es egal, aus welchem Verein die Person kommt.	

\clearpage

\section{Zusammenfassung}
Meine Hauptziele kann man wie folgt zusammenfassen. Schach ist schon Lange kein Hobby, welches im Hinterzimmer stattfindet. Jeder hat, kann und sollte Zugang zu Schach finden. Egal ob Schach als Hobby oder als Sport veranstaltet wird. Ich finde, dass die ThSj ist eine geeignete Institution, um dieses zu vermitteln. Dennoch muss der Anschluss gefunden werden. Schach im Internet darf nicht als Gefahr angesehen werden, sondern als Hilfe. Besonders in Krisenzeiten bietet das Internet unendlich viele Möglichkeiten, um Training und Events zu veranstalten. Das Schach in Thüringen steht gerade an einem Wendepunkt und ich danke allen, dass ich dabei sein darf, den zukünftigen Weg mit zu beschreiten. Schach ist für jeden da. Das sollte das Motto der aktuellen Stunde sein.


\vspace{1.5cm}

Ich hoffe auf eine gute Zusammenarbeit.

\vspace{2.25cm}
Mit freundlichen Grüßen: \\ 
\vspace{2cm}
% Die Unterschrift liegt aus Sicherheitsgründen, wo anders.
%\includegraphics[width=0.25\textwidth]{Unterschrift.png}

	


\end{document}
