\chapter{Training 08.12.2020}
 Die folgenden Aufgaben beziehen sich auf die erste Lichess Studie \cite{Study1}.
 
\section{Kapitel 1}
\subsection{Mit Schwarz}

\black{8/3k4/8/3K4/3P4/8/8/8 b - - 0 1}


\links
Der weiße Bauer auf dem Feld b4 hat die Mittellinie noch nicht überschritten. \\
Die Schlüsselfelder des weißen Bauerns auf b4 sind c6, d6 und e6.\\
 Wenn der weiße König diese erreicht, ist die Partie für weiß gewonnen. Schwarz muss deshalb dies verhindern. Dies ist ihm aber nicht möglich. \\
 Sobald der König auf e7 zieht deckt er das Schlüsselfeld c6 nicht mehr. \\  Folglich kann schwarz nicht verhindern, dass Weiß eines der begehrten Felder erreicht. Demnach verliert er die Partie

\pagebreak

\subsection{Mit Weiß}
\white{8/4k3/8/3K4/3P4/8/8/8 w - - 0 2}


\links
Nach Ke7 gibt dieser das Schlüsselfeld c6 frei. Dieses kann Weiß nun besetzen. Dadurch hat er das Schlüsselfeld erreicht und gewinnt die Partie. \\

\centering
\begin{tabular}[h]{c|c}
	\textbf{Weiß}  & \textbf{Schwarz} \\
	\hline
	Kc6 & Kd8 \\
	d5 & \unterstreichen{1-0}
\end{tabular}


\section{Kapitel 3}

\white{k7/8/2P5/8/K7/8/8/8 w - - 0 1}


\links
Weiß möchte in dieser Partie auf Gewinn spielen. \\
Der Bauer hat die Mittellinie bereits überschritten, daher sind die Schlüsselfelder die Matrix von b7 bis d8. Um diese zu erreichen, spielt Weiß hier Kb5, denn er möchte um den Bauern herumlaufen. Die Opposition braucht weiß nicht zu fürchten, da der weiße Bauer das Feld b7 deckt. Nach Kb8 kann Weiß mit dem König nach b6 ziehen. Um an dem Bauern zu bleiben zieht Schwarz Kc8. Daraufhin kann ich c7 spielen und die Partie mit einem Punkt für mich beenden. \\
Wenn der Bauer den König mit Schach zwingt unter ihn zu ziehen, endet die Partie unentschieden. Zwingt der Bauer den König ohne Schach unter den Bauern, endet die Partie gewonnen. So auch hier. \\

\centering
\begin{tabular}[h]{c|c}
	\textbf{Weiß}  & \textbf{Schwarz} \\
	\hline
	Kb5 & Kb8\\
	c7 & \unterstreichen{1-0}
\end{tabular}


\section{Kapitel 4}
\black{8/k7/p1K5/8/1P6/8/8/8 b - - 0 1}


\links
Hier ist schwarz am Zug.\\
Dieser ist offensichtlich mit einem Remis glücklich. Als mögliche Züge kommt hier nur a5 in Frage. Nun hat Weiß 2 Optionen. Entweder er schlägt diesen oder er ignoriert ihn. Wenn er ihn schlägt, ist es ein Randbauer und der schwarze König hat das Eckfeld. Also spielt Weiß b5.\\
Schwarz möchte nun den Bauern auf der B-Linie blockieren. Dazu muss er diese betreten. Dies geht nur auf dem Feld b8. Wenn er dies nicht jetzt tut, kann er es gar nicht mehr tun, denn Weiß riegelt das Feld sonst mit Kc7 ab.
Daher zieht schwarz Kb8. \\
Weiß muss sich jetzt um dem a Bauern kümmern und geht über c5-b4 an den laufenden Bauern heran und schlägt diesen. Der weiße König steht nun auf a3.\\ 
schwarz kann nun ebenfalls an den Bauern herranlaufen. Nun stehen sich die Beiden Könige in einer Opposition gegenüber. Daher endet diese Partie unentschieden. \\
\centering
\begin{tabular}[h]{c|c}
	\textbf{Weiß}  & \textbf{Schwarz} \\
	\hline
	$\cdots$ & a5\\
	b5 & Kb8 \\
	Kc5 & a4 \\
	Kb4 & a3 \\
	Kxa3 & Kb7 \\
	Kb4 & Kb6 \\
	\unterstreichen{\Remis}
\end{tabular}

\section{Kapitel 5}
\white{6k1/8/3K2p1/8/6P1/8/8/8 w - - 0 1}\\
\links
In dieser Stellung ist zu erkennen, dass der Bauer auf g4 markant steht. Nach dem Zug g5 sperrt dieser Bauer dem schwarzen König die wichtigen Fluchtfelder h6 und f6 ab. Der weiße König bedrängt ihn zusätzlich von der d bzw. f Linie. Dadurch muss der schwarze  König langfristig den Bauern auf g6 aufgeben.\\
\centering
\begin{tabular}[h]{c|c}
	\textbf{Weiß}  & \textbf{Schwarz} \\
	\hline
	g5 & Kf7 \\
	Kd7 & Kg7 \\
	Ke7 & Kh7 \\
	Kf7 & Kh8 \\
	Kxg6 & Kg8 \\
	Kh6 & Kh8 \\
	g6 & Kg8 \\
	g7 & $\cdots$\\
	\unterstreichen{1-0}
\end{tabular}

\section{Kapitel 6}
\white{8/ppp5/8/PPP5/8/7k/8/7K b - - 0 1}\\
\links

Hier wird recht schnell deutlich, dass man einen Durchbruch erzeugen muss. Daher weis ich, dass ich mit b6 anfangen muss. Wenn nun einer der beiden benachbarten Bauern von schwarz den b6 schlägt, kann ich mir einen Freibauern bilden. Dieser läuft dann durch und kann sich umwandeln. \\
\centering
\begin{tabular}[h]{c|c}
	\textbf{Weiß}  & \textbf{Schwarz} \\
	\hline
	b6 & cxb6\\
	a6 & bxa6\\
	c6 & a5 \\
	c7 & Kh4\\
	c8D & $\cdots$ \\
	\unterstreichen{1-0}
\end{tabular}
\clearpage

\section{Kapitel 7}
\black{r3r1k1/5ppp/q7/8/1B2n3/R1P4Q/5PPP/5RK1 b - - 0 1}\\
\links

In dieser Matt-Aufgabe fällt auf das schwarz viele aktive Figuren am Königsflügel hat. Ferner fällt der Turm auf e8 auf, welcher die gesamte e-Linie komplett kontrolliert.  Nun gilt es nur noch den Schlag zu finden.\\
\centering

\begin{tabular}[h]{c|c}
	\textbf{Weiß}  & \textbf{Schwarz} \\
	\hline
	$\cdots$ & Dxf1+ \\
	Kxf1 & Sd2+ \\
	Kg1 & Ke1\#\\
	$\cdot$ & \unterstreichen{0-1}
\end{tabular}

\section{Kapitel 8}
\white{3r3k/pbq1b1p1/1p3n1p/5Q1P/3n3N/P7/BP3PP1/2B1R2K w - - 0 1}\\
\links

Hier brauche ich nicht viel sagen. Der Springer lacht doch schon. Mit dem Läufer auf a2 schreit das schon nach einem erstickten Matt. \\
\centering
\begin{tabular}[h]{c|c}
	\textbf{Weiß}  & \textbf{Schwarz} \\
	\hline
	Sg6+ & Kh7\\
	Sf8+ & Kh8 \\
	Dh7+ & Sxh7\\
	Sg6\# & $\cdot$\\
	\unterstreichen{1-0}
\end{tabular}

\section{Kapitel 9}
\black{k3r2r/pp4bp/6p1/5p2/N2P4/8/1P3PPP/R1R3K1 w - - 0 1}\\
\links

Diese Aufgabe ist auch recht bekannt. Vor allem wenn man sich die a-Linie betrachtet. Auch hier scheint der Springer schon ausdrücklich nach b6 zu wollen. Der Rest ergibt sich dann.\\
\centering

\begin{tabular}[h]{c|c}
	\textbf{Weiß}  & \textbf{Schwarz} \\
	\hline
	Sb6+ & Kb8 \\
	Sd7+ & Ka8 \\
	Txa7+ & Kxa7 \\
	Ta1\# & $\cdot$\\
	\unterstreichen{1-0}
\end{tabular}

\section{Kapitel 10}
\white{8/p2p4/8/8/8/k7/5P1P/7K w - - 0 1}\\
\links

Weiß kann nicht blind mit dem f oder h Bauern loslaufen. Das liegt daran, dass der König den f Bauern einholen würde und mit d5-d1 genauso schnell ist wie weiß. \\ 
Allerdings kann der König nicht den h Bauern einholen. Interessant ist auch, dass in aktueller Stellung der f-Bauer mit Schach einzieht.\\
Wir spielen nun f4 und immer wenn der König in Richtung der Bauern zieht, ziehen wir den H Bauern. Dadurch können wir unseren f-Bauern mit Schach umwandeln. Der schwarze König schlägt diesen und steht nun ebenfalls auf der achten Reihe. Dadurch können wir mit h8D+ uns eine neue Dame holen und den d-Bauern daran hindern sich umzuwandeln. \\
\centering
\begin{tabular}[h]{c|c}
	\textbf{Weiß}  & \textbf{Schwarz} \\
	\hline
	f4 & Kb4 \\
	h4 & d5 \\
	f5 & Kc5 \\
	h5 & d4 \\
	f6 & Kd6 \\
	h6 & d3 \\
	f7 & Ke7 \\
	h7 & d2 \\
	f8D+ & Kxf8 \\
	h8D+ & $\cdot$\\
	\unterstreichen{1-0}
\end{tabular}

\section{Kapitel 11}
\white{1r3k2/2q2b1p/5Qp1/1p2n3/p3P3/8/PPP3PP/1K1R1R2 w - - 0 1}\\
\links

In dieser Aufgabe ist von vorne nicht zu erkennen worum es eigentlich geht. \\ 
Das Matt mit Dxf7 wird leider von dem Springer und der Dame verhindert. Folglich sollte man schauen ob man diese Ablenken kann.  Nach Td7 muss Dxd7 kommen (Sxd7 $\rightarrow$ Dxf7\#). Dieser Zug lenkt aber die Dame von einem anderen Feld ab; und zwar von dem Turm auf b8. Dieser ist nun ungedeckt und kann mit Dh8+ angegriffen werden. Den Rest erkennt man in der Partie.\\
\centering
\begin{tabular}[h]{c|c}
	\textbf{Weiß}  & \textbf{Schwarz} \\
	\hline
	Td7 & Dxd7 \\
	Dh8+ & Ke7 \\
	Dxe5+ & De6 \\
	Dc7+ & Dd7 \\
	Dxb8 & $\cdot$\\
	\unterstreichen{1-0}
\end{tabular}

