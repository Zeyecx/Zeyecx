\chapter{Training 15.12.2020}
 Die folgenden Aufgaben beziehen sich auf die zweite Lichess Studie \cite{Study2}

\section{Kapitel 1}
\links
Diese Aufgabe ist eine Wiederholung. \\

\subsection{Mit Schwarz}
\black{8/3k4/8/3K4/3P4/8/8/8 b - - 0 1}\\

\links
Wenn in dieser Stellung schwarz am Zug ist, gewinnt Weiß. Das liegt daran, dass er nach Ke7 das Schlüsselfeld c6 freigibt und nach Kc7 das Schlüsselfeld e6. \\
\centering
\begin{tabular}[h]{c|c}
	\textbf{Weiß}  & \textbf{Schwarz} \\
	\hline
	$\cdots$ & Ke7 \\
	Kc6 & Kd8 \\
	Kd6 & Kc8 \\
	Ke7 & kc7 \\
	d5 & $\cdot$ \\
	\unterstreichen{1-0}
\end{tabular}

\subsection{Mit Weiß }
\white{8/3k4/8/3K4/3P4/8/8/8 b - - 0 1}

Wenn Weiß in dieser Stellung am Zug ist, endet die Partie \unterstreichen{\texttt{Remis}}, da schwarz immer alle Schlüsselfelder abdecken kann.

\section{Kapitel 2}
\black{8/8/8/6p1/7k/8/6K1/8 b - - 0 1}\\
\links

In dieser Stellung gibt es nur 2 Züge die interessant sind:
 \begin{itemize}
	\item Kg4 
	\item g4
\end{itemize}
Nach g4 ist die Partie leider nach der Definition aus 1.1.2 Remis. Deswegen muss hier Kg4 kommen und dann ist das eine Frage des Zugzwangs. Denn nach Kg4 bekomme ich immer ein Schlüsselfeld.\\
\centering
\begin{tabular}[h]{c|c}
	\textbf{Weiß}  & \textbf{Schwarz} \\
	\hline
	$\cdots$ & Kg4 \\
	Kh2 & Kf3 \\
	Kh3 & g4+ \\
	Kh2 & g3+ \\
	Kh1 & Kf2 \\
	$\cdot$ & \unterstreichen{0-1}
\end{tabular}	

\section{Kapitel 3}
\black{8/8/8/8/8/7p/7k/5K2 b - - 0 1}\\
\links
Diese Stellung ist \unterstreichen{\texttt{Remis}}, da Weiß immer das Feld h2 decken kann. Schwarz kann dagegen nichts machen. \\
\centering
\begin{tabular}[h]{c|c}
	\textbf{Weiß}  & \textbf{Schwarz} \\
	\hline
	$\cdots$ & Kg3 \\
	Kg1 & h2+\\
	Kh1 & Kf2 \\
	$\cdot$ & \unterstreichen{\Remis}
\end{tabular}	

\section{Kapitel 4}
\white{8/2k5/3p4/1K1P4/8/8/8/8 w - - 0 1}\\
\links
Hier ist die Idee, dass ich mit dem König von \Kommentar{links} an den König herranlaufe und diesen so abdränge. Die beiden Widderbauern decken jeweils die Felder neben den des anderen Bauerns. Dies ist der Witz der Aufgabe, da ich so den König von dem Bauern abdrängen kann. \\
\centering
\begin{tabular}[h]{c|c}
	\textbf{Weiß}  & \textbf{Schwarz} \\
	\hline
	Ka6 & Kd7 \\
	Kb7 & Ke7 \\
	Kc7 & Ke8 \\
	Kxd6 & Kd8 \\
	Ke6 & Ke8 \\
	d6 & Kd8 \\
	d7 & $\cdot$ \\
	\unterstreichen{1-0}
\end{tabular}

\section{Kapitel 5}
\black{8/8/8/8/1P6/6k1/8/6K1 b - - 0 1}\\
\links
 Die schöne \Kommentar{Quadratregel}. \\
 Das Quadrat des Bauerns geht von b4-b8-f4-f8. Da schwarz dran ist, kann er dieses Quadrat betreten und so den Bauern in der Umwandlung schlagen.\\
 \centering
 \begin{tabular}[h]{c|c}
 	\textbf{Weiß}  & \textbf{Schwarz} \\
 	\hline
 	$\cdots$ & Kf4 \\
 	b5 & Ke5 \\
 	b6 & Kd6 \\
 	b7 & Kc7 \\
 	\unterstreichen{\Remis} & \unterstreichen{\Remis}
 \end{tabular}

\section{Kapitel 6}
\white{8/5p2/4p3/8/3P4/5k2/P7/5K2 w - - 0 1}\\
\links
Hier geht das Quadrat des a2 Bauerns über folgende Felder:  a2-g2-a8-g8. \\
Dadurch würde der Schwarze König den Bauern gewinnen können. Durch den Zug g5 stelle ich einen schwarzen Bauern auf das Feld. \\
\centering
\begin{tabular}[h]{c|c}
	\textbf{Weiß}  & \textbf{Schwarz} \\
	\hline
	d5 & exd5 \\ 
	a4 & Ke4 \\
	a5 & $\cdot$ \\
	\unterstreichen{1-0}
\end{tabular}

\section{Kapitel 7}
\white{8/8/8/3kp3/8/8/8/3K4 w - - 0 1}\\
\links

Der Bauer hat die Mitte bereits übertreten, dadurch sind die Schlüsselfelder d3-d2-f2-f3.\\
Weiß muss erreichen, das er die Oposition erzwingt. Sollte er direkt auf die zweite Reihe gehen, ist dies nicht möglich. Daher zieht er erst Ke1. Nun muss schwarz dies erlauben.\\

\centering
\begin{tabular}[h]{c|c}
	\textbf{Weiß}  & \textbf{Schwarz} \\
	\hline
	Ke1 & Ke4 \\
	Ke2 & $\cdot$ \\
	\unterstreichen{1-0}
\end{tabular}

\section{Kapitel 8}
\white{5k2/8/8/8/1P6/8/8/3K4 w - - 0 1}\\
\links
Weiß muss offensichtlich an die Bauern heranlaufen. Dies ist auf 2 Feldern möglich. c4 und a4. Für das Feld benötigt er 3 Halbzüge und schwarz mehr. Das bedeutet er ist schneller da.\\
Dies stellt den Zeitvorteil dar. \\
\centering
\begin{tabular}[h]{c|c}
	\textbf{Weiß}  & \textbf{Schwarz} \\
	\hline
	Kc2 & Ke8 \\
	Kb3 & Kd8 \\
	Ka4 & Kc8 \\
	Ka5 & Kb8 \\
	Kb6 & $\cdot$ \\
	\unterstreichen{1-0}
\end{tabular}

\section{Kapitel 10}
\white{k7/8/1p6/8/1P6/5K2/8/8 w - - 0 1}\\
\links

Die Aufgabe ist eine Wiederholung der Aufgabe 1.5 . Nach b5 kann ich den König wieder erfolgreich abdrängen.  \\
\centering
\begin{tabular}[h]{c|c}
	\textbf{Weiß}  & \textbf{Schwarz} \\
	\hline
	b5 & Kb7 \\
	Ke3 & $\cdot$ \\
	\unterstreichen{1-0}
\end{tabular}

\section{Kapitel 11}
\white{8/7p/8/4kpPP/8/8/8/5K2 w - - 0 1}\\
\links

Die Idee ist, dass schwarz nicht mit dem König von unten an die Bauern herankommt, da diese sonst durchlaufen. Wenn ich mit dem König ziehe, kommt sofort f4 und ich kann meine Bauern nicht mehr verteidigen (Remis). \\
Wenn ich direkt mit g6 loslaufe holt mich der König dennoch ein. Folglich bleibt mir nur der Zug h6 mit dem Opfer auf g6. Grundlegend ist es das selbe, bloß das der h Bauern nun bereits ein Feld weiter vorn steht und sich so umwandeln kann. 

\centering
\begin{tabular}[h]{c|c}
	\textbf{Weiß}  & \textbf{Schwarz} \\
	\hline
	h6 & Ke6 \\
	g6 & $\cdot$ \\
	\unterstreichen{1-0}
\end{tabular}

\section{Kapitel 12}
\white{8/8/3p4/k7/P1P5/p7/2K5/8 w - - 0 1}\\
\links

Offensichtlich steht weiß hier etwas seltsam. Der Bauer auf a3 ist fast durch und die eigenen Bauern stehen irgendwie fremdartig.  Dennoch ist der erste Zug recht offensichtlich. \\
Kb3. Wenn schwarz nun b2 spielt, schlage ich diesen nicht raus sondern spiele Kb2. Das liegt daran, dass ich dann kein Zugzwang wegen dem c Bauer habe. Erst wenn schwarz jetzt den Bauern umwandelt tausche ich diesen. 
Daraufhin nimmt schwarz den a4 und gibt mir so die Zeit c5 zu spielen um die Partie Remis zu halten. \\
Denn nach c5 dxc5 ist es wieder eine definierte Stellung.\\
\centering

\begin{tabular}[h]{c|c}
	\textbf{Weiß}  & \textbf{Schwarz} \\
	\hline
	Kb3 & a2 \\
	Kb2 & a1D+ \\
	Kxa1 & Kxa4 \\
	c5 & dxc5\\
	Ka2 & $\cdot$ \\
	\unterstreichen{\Remis} & \unterstreichen{\Remis}
\end{tabular}

\section{Kaptiel 13}

\white{8/8/3p4/2k5/4P3/8/8/1K6 w - - 0 1}

Der erste Zug e5 stammt aus vorherigen Partien. Nun geht es darum nicht in eine Opposition zu gelangen. Dies ist nur auf dem Feld Kc1 möglich. Sollte schwarz hier die Opposition aufstellen verliert Weiß, sonst ist es Remis. \\
\centering
\begin{tabular}[h]{c|c}
	\textbf{Weiß}  & \textbf{Schwarz} \\
	\hline
	e5 & dxe5\\
	Kc1 & Kc4 \\ 
	Kc2 & $\cdot$ \\
	\unterstreichen{\Remis} & \unterstreichen{\Remis}
\end{tabular}


