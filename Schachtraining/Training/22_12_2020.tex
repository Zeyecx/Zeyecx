\chapter{Training 22.12.2020}
 Die folgenden Aufgaben beziehen sich auf die zweite Lichess Studie \cite{Study2}
 
 
\section{Kapitel 14}
\white{8/1k6/1p6/1K6/2P5/8/P7/8 w - - 0 1}\\
\links

Man könnte überlegen, mit a4 loszulaufen nur um dann festzustellen, dass der König jedes mal im Weg steht. \\
Die Idee ist also mit a3 dem schwarzen ein Tempo zu geben um so das Feld a6 zu bekommen. Dann können beide Partein um den Bauern herumlaufen und der c4 Bauer sorgt dann dafür, dass schwarz den Bauern aufgeben muss.\\
 \centering
\begin{tabular}[h]{c|c}
	\textbf{Weiß}  & \textbf{Schwarz} \\
	\hline
	a3 & Kc7 \\
	Ka6 & Kc6 \\
	a4 & Kc7 \\
	Ka7 & Kc6 \\
	Kb8 & Kd6 \\
	Kb7 & Kc5 \\
	Kc7 & Kxc4 \\
	Kxb6 & $\cdot$ \\
	\unterstreichen{1-0}
\end{tabular}

\section{Kapitel 15}
\white{4k3/8/8/1p5p/1P5P/8/8/4K3 w - - 0 1}\\
\links
Beide Könige laufen sofort auf die vierte bzw sechste Reihe los. Dort stehen sie sich in einer Opposition gegenüber und schwarz muss ziehen. \\
Weiß kann mitziehen. Wenn schwarz sich nah an einem Bauern befindet, kann dieser nicht weiter und muss den Bauern aufgeben. \\ Er bekommt zwar den Bauern auf der anderen Seite. Allerdings muss er dahin noch laufen. \\
\centering
\begin{tabular}[h]{c|c}
	\textbf{Weiß}  & \textbf{Schwarz} \\
	\hline
	Ke2 & Ke7 \\
	Ke3 & Ke6 \\
	Ke4 & Kd6 \\
	Kd4 & Ke6 \\
	Kc5 & $\cdot$\\
	\unterstreichen{1-0}
\end{tabular}



\section{Kapitel 16}
\white{8/8/8/4p1p1/8/5P2/6K1/3k4 w - - 0 1}\\
\links

Es darf keine Stellung entstehen wo der weiße König auf g3 und der schwarze auf e3 steht (und weiß dran ist). So eine Stellung verliert weiß immer.\\
Deswegen muss Weiß versuchen immer in Opposition zu gehen, damit dieser Fall nicht eintritt. \\
\centering
\begin{tabular}[h]{c|c}
	\textbf{Weiß}  & \textbf{Schwarz} \\
	\hline
	Kh1 & Ke1 \\
	Kg1 & Ke2 \\
	Kg2 & Ke3 \\
	Kg3 & Kd2 \\
	Kh2 & Kd3 \\
	Kh3 & $\cdot$ \\
	\unterstreichen{1-0}
\end{tabular}

\section{Kapitel 17}
\white{8/3k2p1/8/5P2/8/8/8/1K6 w - - 0 1}\\
\links

Hier ist es wichtig zu erkennen, wann ich den Hebel f6 spiele. \\
Wenn ich gleich f6 spiele, kommt der Bauer zwar auf die f-Linie, allerdings habe ich dann keine Möglichkeit mehr mit dem schwarzen König in Opposition zu gehen. Deswegen muss ich mit Kc2 anfangen.\\
Danach geht alles wie gewohnt.\\

\centering
\begin{tabular}[h]{c|c}
	\textbf{Weiß}  & \textbf{Schwarz} \\
	\hline
	Kc2 & Kd6 \\
	f6 & gxf6 \\
	Kd2 & Ke6 \\
	Ke2 & f5\\
	Kf3 & Ke5 \\
	Ke3 & $\cdot$ \\
	\unterstreichen{\Remis} & \unterstreichen{\Remis}
\end{tabular}



\section{Kapitel 18}
\black{8/5p2/8/6Pk/5P2/8/8/7K w - - 0 1}\\
\links

Wenn ich g6 spiele, muss entweder der König oder der Bauer nehmen. Das selbe gilt für das selbige f5. \\
Danach kann ich mithilfe von Kf1 die Opposition halten.\\

\centering
\begin{tabular}[h]{c|c}
	\textbf{Weiß}  & \textbf{Schwarz} \\
	\hline
	Kc2 & Kd6 \\
	f6 & gxf6 \\
	Kd2 & Ke6 \\
	Ke2 & f5 \\
	Kf3 & Ke5 \\
	Ke3 & $\cdot$ \\
	\unterstreichen{\Remis} & \unterstreichen{\Remis}
\end{tabular}


\section{Kapitel 19}
\white{8/6k1/3K1p2/4p1p1/4P1P1/5P2/8/8 b - - 0 1}\\
\links
Auch hier gilt, Oppositionen sind nice.\\
Also bleibt hier nur das Feld Kh6. Und dann einfach mitziehen. \\

\centering
\begin{tabular}[h]{c|c}
	\textbf{Weiß}  & \textbf{Schwarz} \\
	\hline
	$\cdots$ & Kh6\\
	f6 & gxf6 \\
	Ke6 & Kg6 \\
	Ke7 & Kg7 \\
	\unterstreichen{\Remis} & \unterstreichen{\Remis}
\end{tabular}
